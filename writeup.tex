% Security of mixnets with a cascade topology.

\documentclass[letter]{article}
\usepackage{color,soul}
\usepackage[dvipsnames]{xcolor}
\usepackage[utf8]{inputenc}
\usepackage[margin=4cm]{geometry}
\usepackage{graphics}
\usepackage[toc,page]{appendix}
\usepackage[font={small}]{caption}
\usepackage{hyperref}
\usepackage{amsmath}
\usepackage{amsfonts}
\usepackage{parskip}
\usepackage{framed}
\usepackage{multicol}

\hypersetup{
    colorlinks,%
    citecolor=black,%
    filecolor=black,%
    linkcolor=black,%
    urlcolor=black
}

\def\dashuline{\bgroup
  \ifdim\ULdepth=\maxdimen  % Set depth based on font, if not set already
	  \settodepth\ULdepth{(j}\advance\ULdepth.4pt\fi
  \markoverwith{\kern.15em
	\vtop{\kern\ULdepth \hrule width .3em}%
	\kern.15em}\ULon}

\newcounter{foot}
\setcounter{foot}{1}
\setlength\parindent{2em}


%%%%% Macros. %%%%%%%%%%%%%%%%%%%%%%%%%%%%%%%%%%%%%%%%%%%%%%%%%%%%%%%%%%%%%%%%%

\DeclareMathAlphabet\mathbfcal{OMS}{cmsy}{b}{n}
\newcommand{\note}[1]{}

% Sets
\newcommand{\Z}{\mathbb{Z}}
\newcommand{\Zmodp}{\mathbb{Z}_p}
\newcommand{\Zmodq}{\mathbb{Z}_q}
\newcommand{\N}{\mathbb{N}}
\newcommand{\bits}{\{0,1\}}
\newcommand*\Union{\bigcup}
\newcommand*\Intersect{\bigcap}
\newcommand*\union{\cup}
\newcommand*\intersect{\cap}
\newcommand*\by{\times}
%\newcommand{\sample}{\xleftarrow{\text{\tiny{\$}}}}
\newcommand{\sample}{\mathrel{\leftarrow\mkern-14mu\leftarrow}}

% Asymptotics
\newcommand{\negl}{\proc{negl}}
\newcommand{\poly}{\proc{poly}}

\newcommand{\heading}[1]{\noindent \textsc{#1}}

% Misc.
\newcommand{\brackets}[1]{\langle #1 \rangle}
\newcommand{\view}{\proc{view}}

\newcommand{\pk}{K}
\newcommand{\sk}{k}

\newcommand{\advA}{\mathcal{A}}

%%%%% Title. %%%%%%%%%%%%%%%%%%%%%%%%%%%%%%%%%%%%%%%%%%%%%%%%%%%%%%%%%%%%%%%%%

\date{\today}
\title{Verifying a shuffle in zero-knowledge}
  \author{Chris Patton}

\setcounter{tocdepth}{2}

\renewcommand{\cal}[1]{\mathcal{#1}}

%%%%% Content. %%%%%%%%%%%%%%%%%%%%%%%%%%%%%%%%%%%%%%%%%%%%%%%%%%%%%%%%%%%%%%%%

\begin{document}

  \maketitle

\section{Introduction}

A mixnet (sometimes called a ``mix-network'') is a service designed to
facilitate anonymous communications over the internet \cite{chaum81}. Each of a
set of users $\cal{Q} = \{P_1, \dots, P_n\}$ wish to send a message to a user in
$\cal{Q}$. Each encrypts their message $M$ and the identity of the intended
recipient $P_i$ under the public key of the mixnet server (heereafter referred
to as the ``mix'') and transmits the ciphertext.  The mix server waits until it
receives $m \le n$ ciphertexts from $m$ distinct users.  It then decrypts each
message and destination pair under its secret key and transmits each message to
its intended destination. Crucially, the mix transmits the messages in a random
order so as to hide the correspondence between senders and recipients.  As long
as the mix is trusted and the user's ciphertexts are faithfully transmitted to
the mix, this simple approach suffices to preserve anonymity of its
participants.

Normally we think of the mixnet being composed of a cascade of mix servers, each
decrypting, shuffling, then transmitting to the next mix in the cascade until
finally the last server outputs the users' messages. This is meant to distribute
trust so that as long as the messages pass through at least one honest mix,
anonymity is achieved. Still, the trust model here is rather strong. For
example, the mix could send a fake message on behalf of one of the senders, or
redirect it to a different recipient.  We consider the problem of publically
verifying that the outputs are indeed a permutation of the inputs. Moreover, we
require that the correspondence not be revealed. Andrew Neff describes in
\cite{neff01} a method for proving a shuffle in zero-knowledge.

Neff's solution is based on the ElGamal public key encryption scheme, which is
known to be semantically secure assuming the discrete log problem (decisional
Diffie-Hellman, or DDH) is hard \cite{elgamal84}. Based on this assumption, Neff
proves his protocol achieves a property known as honest verifier zero-knowledge
(HVZK) \cite{goldreich}. Informally speaking, this means that as long as the verifier
faithfully executes the protocol, the mix can prove the outputs are a
permutation of the inputs without revealing the permutation.

I formulate and define security for the problem and present Neff's protocol.
For simplicity, I limit the discussion to a mixnet with a single mix server. In
this scenario, users encrypt their messages under the public key of the mix. The
mix waits until it receives a full batch of ciphertexts, decrypts them under its
private key, and shuffles the output. In section~\ref{section:prelim}, I specify
ElGamal encryption, describe the DDH assumption, and sketch the HVZK notion. In
section~\ref{section:neff-shuffle}, I present the protocol and sketch its
security. I implement the general $k$-shuffle as described in this section in
the Go programming language.\footnote{Freely available at {\it
github.com/cjpatton/shuffle}}.

\section{Preliminaries} \label{section:prelim}

If $X$ is a finite set, let $x \sample X$ denote uniformly sampling an
element $x$ from the set. Let $y \gets A(x)$ denote running $A$ on input $x$ and
assigning its output to $y$. Let $y \gets A(x; r)$ denote the execution of a
probabilistic algorithm with the sequence of coins $r \in \bits^\infty$. Let $y
\sample A(x)$ denote choosing $r \sample \bits^\infty$ and executing $y \gets
A(x; r)$. Let $[i..j]$ where $i \le j$ denote the set of integers from $i$ to
$j$ inclusive.
Finally, a function $\Delta(n)$ is negligible if for all positive polynomials
$p(\cdot)$ and sufficiently large $n$, $\Delta(n) < \frac{1}{p(n)}$.

\subsection{ElGamal encryption}

Fix a prime number $p$ and let $\Zmodp$ denote the modular ring of order $p$.
In what follows, we assume arithmetic is performed in the modular ring $\Zmodp$.
Let $g \in \Zmodp$ such that $g^q = 1$ where $q$ is prime. Then $\brackets{g} =
\Zmodq^*$ is a subgroup of $\Zmodp^*$ and $q | (p-1)$. We call $(p, q, g)$ the
\textit{public parameters}. The \textit{secret-key} is chosen uniformly from
the set of powers of $g$ mod $p$: let $\sk \sample [1..q-1]$. Let $\pk = g^\sk$ be
the \textit{public-key}.

Encryption of $m \in \bits^*$ proceeds as follows:
the plaintext is encoded as $M \in \Zmodp^*$. (Of course, only a subset of
$\bits^*$ may be encoded for a fixed modulus $p$.)
Let $r \sample [1..q-1]$,
$R = g^r$,
$S = K^r$, and
$C = M \cdot S$.
The ciphertext is the tuple $(R, C)$.
Decryption of a ciphertext $(R, C) \in \Zmodp^* \by \Zmodp^*$ proceeds as follows:
let $S = R^\sk$ and
$M = C \cdot S^{-1}$.
The plaintext $m$ is recovered by decoding $M$.

This encryption scheme is known to be \textit{semantically secure} under the
decisional Diffie-Hellman assumption \cite{ty98}. Informally speaking, this
means that if given $\pk$, it is infeasible for any adversary to determine
$\sk$, then given $(C,R)$, it is infeasible for any adversary to
learn anything meaningful about $M$ (nor $m$). As we're not
interested in the security of ElGamal from a privacy standpoint, we won't
formalize semantic security. However, we define the DDH assumption.

\subsection{Decisional Diffie-Hellman}

Let $(p, q, g)$ be the public parameters of the ElGamal cryptosystem. The game
the adversary $\advA$ plays is defined as follows: let $a, b, r \sample [1..q-1]$ and
let $A = g^a$, $B = g^b$, $Y_0 = g^r$, and $Y_1 = g^{ab}$. Choose a random bit
$b \sample \bits$. On input $(A, B, Y_b)$, the adversary outputs a bit
$b^\prime \in \bits$. The advantage of $\advA$ is defined as
\[
  \Delta(q) = |\Pr[\advA(A, B, Y_1)=1] - \Pr[\advA(A, B, Y_0)=1]|
\]
The DDH assumption is that $\Delta(q)$ is negligible.



\subsection{Honest verifier zero-knowledge}

\section{The Neff shuffle} \label{section:neff-shuffle}

Describe what the mix does with specified syntax.

\subsection{The basic protocol and its security}
Not useful yet ...

\subsection{Simple $k$-shuffle}
Not useful yet ...

\subsection{General $k$-shuffle}
This is what we actually implement.


\begin{thebibliography}{99}

  \bibitem{chaum81} David Chaum, 1981. ``Untraceable electronic mail, return addresses, and digital pseudonyms.''
    {\tt http://freehaven.net/anonbib/cache/chaum-mix.pdf}

  \bibitem{elgamal84} Taher Elgamal, ``A public key cryptosystem and a signature
    scheme based on discrete logarithms.'' Appeared in \textit{CRYPTO'84.}

  \bibitem{cp92}  D. Chaum and T.P. Pedersen. ``Wallet databases with
    observers.'' Appeared in \textit{CRYPTO'92.}

  \bibitem{ty98} Yiannis Tsiounis and Moti Yung 1998. ``On the security of
    ElGamal based encryption.'' Appeared in \textit{PKC'98.}

  \bibitem{goldreich} Oded Goldreich, 2001. \textit{Foundations of Cryptography.}

  \bibitem{neff01} C. Andrew Neff, 2001. ``A verifiable secret shuffle and its application to e-voting.'' {\tt http://freehaven.net/anonbib/cache/shuffle:ccs01.pdf}

  \bibitem{ai03} Masayuki Abe and Hideki Imai, 2003. ``Flaws in some robust
    optimistic mix-nets.'' Appeared in \textit{ACISP'03.}

\end{thebibliography}

\end{document}
